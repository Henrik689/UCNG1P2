problemanalyse intro

\section{Valg af sensor}
I dette afsnit vil der blive diskuteret forskellige valg af sensorer, deres specifikationer som kan danne grundlag for et valg af en sensor.

Kriterier for valg af sensor er:
\begin{enumerate}
	\item[•]Digital, analog, modstand eller transistor
	    \begin{enumerate}
	        \item[-]Digital er nemmere at connecte i forhold til analog.
	        \item[-]En analog sensor kan blive påvirket af kredsløbet.
	    \end{enumerate}	
	
	\item[•]Præcision på sensoren 
	\item[•]Måleafstand
		\begin{enumerate}
			\item[-]Fra 5 - 20 grader celcius. Reference: Ib Helmer Nielsen
		\end{enumerate}	
	\item[•]Montering
	    \begin{enumerate}
	        \item[-]PCB og rør
	    \end{enumerate}
	\item[•]Pris
	\item[•]Drifts-/ forsynings-spænding 
	    \begin{enumerate}
	        \item[-]UCN Arduino board kompatibel med 3,3 / 5 Volt.
	        \item[-]Arduinoen kan max. måle 3,3 Volt men kan ikke modtage 5 Volt.
	    \end{enumerate}
	
\end{enumerate}


\begin{table}[]
\centering
\begin{tabular}{|l|l|l|l|l|}
\hline
 & LM9523 & LM53DIMA & LM86CIM & Ds18b20 \\ \hline
Digital & Ja & Ja & Ja & Ja \\ \hline
Præcision & $\pm$1 $^{\circ}$C, $\pm$2.5 $^{\circ}$C & $\pm$1 $^{\circ}$C, $\pm$3 $^{\circ}$C & $\pm$1 $^{\circ}$C, $\pm$3 $^{\circ}$C & \begin{tabular}[c]{@{}l@{}}$\pm$0,5$^{\circ}$C (ved -10 \\ til +85$^{\circ}$C)\end{tabular} \\ \hline
Driftstemperatur & 0 $^{\circ}$C til +85 $^{\circ}$C & 0 $^{\circ}$C til 85 $^{\circ}$C & 0 $^{\circ}$C til 85 $^{\circ}$C & -55$^{\circ}$C til 125$^{\circ}$C \\ \hline
Montering & SMD & SMD & SMD & Tht \\ \hline
Pris & 18,10 DKK & 18,85 DKK & 11,864 DKK & 35,20 DKK \\ \hline
Driftsspænding & 3,0 V til 3,6 V & 3,0 V til 3,6 V & 3,0 V til 3,6 V & 3,0 V til 5,5V \\ \hline
\end{tabular}
\caption{Tabel over de forskellige sensorer}
\label{sensor_tabel}
\end{table}



\newpage
