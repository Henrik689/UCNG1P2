\section{Description of the hardware structure and functionality}
In the following section an in depth description of the structure used in designing the hardware and its functionality is presented.

\subsection{UPS}
A UPS (Uninterruptible power supply) is a backup system to the power supply. The UPS is a backup that takes over when the power from the main source fails. The UPS will give you a few minutes to properly shutdown the system, to prevent data loss.
\img{figures/UPSdiagram.png}{Diagram of the UPS function the system}{UPSDiagram}{0.8}\newline
The diagram in fig. \ref{UPSDiagram}.  indicates what function the UPS has in the system (ref. fig \ref{Hardwarediagram}). In case of power failure the UPS will inform the microcontroller over a data connection, which allows the microcontroller to do a proper shutdown.

\subsection{Power supply}
For maximizing the runtime of the surveillance system in the server room, a redundant power supply is selected. 
A redundant power supply contains either two or more  power supplies inside it (ref. figure \ref{RPSU}). Each unit is capable of powering the entire system. Only one unit will run at a time but if it fails, the other takes over. This is called hot swapping. The second power supply is always running on standby, so if the powering unit fails the switch to the second unit will be unnoticeable when the system is running.
\img{figures/RPSU.jpg}{A redundant power supply with two power units}{RPSU}{0.8}\newline
The redundant power supply and the UPS is chosen for their ability to complement each other. The redundant power supply provides insurance against the power supply failing and the UPS provides insurance against power outage.\newline

\subsection{Microcontroller}
The microcontroller is a small computer which comes in a variety of packages and sizes. Compared to a personal  computer that performs a variety of tasks in the background. A microcontroller is typically dedicated to a simple task. 
\img{figures/MCUdiagram.png}{The different components connected to a BUS on a microcontroller}{MCUdiagram}{0.8}\newline
The microcontroller contains a CPU (central processing unit) as the brain of the unit that does all the calculations and logical operations which allows the software to function. The microcontroller also contains memory, which allows the microcontroller to store information so it can be used at a later time. Together with the CPU and the memory is the system clock. The system clock can be thought of as the engine to the microcontroller unit. Without it nothing will work. It is also a system clock which determines what speed the microcontroller will run. 
Communication with the microcontroller is done through the I/O (Input/output) which interconnects with the BUS. The BUS is a subsystem used to connect the different components and transfer data between them. A communication device can be attached to either the I/O or connected directly to the BUS (ref. fig. \ref{MCUdiagram}). 
To meet the hardware requirement specification for a microcontroller (hardware requirement p. 7), a real time microcontroller i chosen. 
A real time microcontroller is defined by measuring in real time, which means the time is precise. That also means when looking through previous logged data, it creates an opportunity to see precisely where i.e. spikes in power consumption has happened or exactly when the spikes occur.

\subsection{Sensors}
\subsubsection{Air Condition}
The A/C (air conditioner) is not a part of our product solution, but a lot of server rooms has air conditioners and this opens up for a dynamic regulated system where it’s possible to adjust the temperature and humidity in the server room.\\
The A/C is a system that controls the temperature,  the humidity, and the airflow.
If the temperature in the room gets above the desired threshold, it can transfer the air outside the server room.
The A/C can cool the room down to a desired temperature which the customer can regulate and is able to control. The A/C will control the airflow in a way to prevent i.e. humidity in the server which can result in a short circuit.

\subsubsection{Humidity}
Normally when having an air conditioner a humidity sensor is not needed, but it is chosen so the humidity can be logged and the A/C’s effect can be monitored.

\subsubsection{Temperature}
The temperature sensor will measure the temperature in the server room. Temperature sensors will be placed in different locations within the room to measure the different values. These values are defined through measurement of the room and the servers in conjunction with the client.

\subsubsection{Smoke}
Smoke is generally a thing you don’t want inside a server room, it can be from burning electronics, or any kind of fire. For this reason it is essential to have a smoke detector in the server room.
The smoke sensor measures through two different parameters. It senses the visible particle change in the air density as well as the invisible. 

\subsubsection{Flood sensor}
The flood sensor consists of a cable that relies on electrical conductivity from water, this means the flood sensor constantly checks the resistance. If the resistances drops the sensor will go off and alarm the system admin.

\subsubsection{Power consumption}
To measure instability in the power consumption a sensor is needed. The sensor measures the consumption of the server clusters. The values measured indicates when and if spikes occur in the main power grid. It is therefore essential to be able to monitor the power grid to prevent data loss. 

\subsubsection{Room access}
To know when someone enters the server room, there will be a magnetic switch connected to the door and a reed switch to the doorway. A reed switch is an electrical switch operated by a magnetic field. 

\subsubsection{Video Surveillance}
The video surveillance is a basic function to add to a server room, it will monitor the server room both inside and outside. The surveillance camera outside will be active at all times, whereas the inside camera will be activated when someone accesses the room.