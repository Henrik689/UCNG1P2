%Opgavebeskrivelse
I denne opgave var formålet at designe, implementere og teste et elektronisk system med henblik på overvågning af vandspild i private husstande.
\\ 
\\
%Overblik
Løsningen er designet omkring et Arduino board, en ds18b20 sensor til måling af temperatur på røret og en dht11 sensor til måling af temperatur og fugtighed i  rummet. Softwaren kan deles op i tre dele, et bibliotek til anvendelse af sensor ds18b20 som blev skrevet ud fra det tilhørende datablad, sensor dht11  og et hovedprogram til styring af funktioner.            
\\
\\
%Fremgangsmåde
Gruppen har ved hjælp af et Arduino board og to sensorer implementeret et stykke software, som blev bygget fra bunden. Funktionerne konstrueret i softwaren udfører to temperatur målinger. Herfra foretages en differens af målingerne og giver et output til en terminal. Denne difference i temperaturen afgører om vandspild forekommer.
\\
\\
%Videreudvikling
Løsningsforslaget blev udvidet, men pga. tidsmangel er dette skrevet som et videreudviklings afsnit i stedet for at stå i design. 
%Data logning -hjemmeside - shield
For at logge data er en hjemmeside blevet konstrueret, hvor Arduinoen kan sende data dertil ved hjælp af et ethernet shield.
%radio modul - Mesh netværk
Gruppen har tilføjet radio moduler og sat det op i  et mesh netwærk.



