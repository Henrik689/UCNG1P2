\section{Sensor}
Sensoren, som bliver beskrevet i hardwareafsnittet, skal kommunikere med arduino boardet. Dette kræver nogle trin som kan findes i databladet til sensoren. På figur \ref{sensor_kom} ses et overblik over funktioner gruppen har lavet, som skal til for at aflæse sensoren. Disse funktioner er konstrueret ud fra databladet, hvor det ses at tre trin skal følges til præcist for at tilgå sensoren. De tre trin er en initialisering(initialization), ROM command og function command.
\\
\\
I databladet ses et flowchart over hvordan mikroprocessoren opnår adgang til sensoren kan tilgås og funktioner den kan udføre. Ud fra dette flowchart har gruppen konstrueret en mindre version, som indeholder de minimale funktioner der skal anvendes for at tilgå sensoren, og aflæse denne. Dette ses på figur \ref{sensor_kom}.

\begin{figure}[h!]
  \centering
  \includegraphics[width=0.8\textwidth]{figures/sensor_communication.png}
  \caption{Kommunikation med sensor.}
  \label{sensor_kom}
\end{figure}
\fxnote{ret størrelse til på alle billeder(noget der skal gøres til sidst!)}

Navne der vises på figur \ref{sensor_kom} er de funktionsnavne der er givet i vores software. De følger flowchartet for det mindste der skal til for at tilgå og aflæse en temperatur måling af sensoren. Måden de er konstureret på vil blive forklaret i følgende underafsnit.


\subsection{Initialize()}
Initialiseringsfunktionen, som ses på figur \ref{sensor_kom}, er konstrueret ved at finde intervallerne som er nødvendige for at tilgå sensoren over en 1-wire forbindelse. Disse er aflæst af en graf fra databladet (jf. figur \ref{sensor_kom}).




\begin{figure}[h!]
  \centering
  \includegraphics[width=0.5\textwidth]{figures/Initialization_timing.png}
  \caption{Fra datablad om hvordan sensor skal initialiseres.}
  \label{sensor_kom}
\end{figure}

Fra databladet ses at 1-wire forbindelsen skal have en reset pulse i minimum 480$\mu$S og en presence pulse i 60 til 240$\mu$S. Dette gøres ved at sende et low signal som reset pulse og derefter sætte forbindelsen i tri-state og pull-up modstanden vil trække signalet højt. 
\\
\\
I koden bliver dette gjort ved at kalde digitalWrite med low som parameter, i kombination med et delay på 500$\mu$S. Derefter sættes pinMode til input og et delay på 500$\mu$S anvendes igen. Dette kan ses på figur \ref{sensor_kode}.

\begin{figure}[h!]
  \centering
  \includegraphics[width=1\textwidth]{figures/Init.png}
  \caption{Initialisering kode.}
  \label{sensor_kode}
\end{figure}

\fxnote{evt noget afslutning på dette underafsnit?}

\subsection{writeByte()}
WriteByte() er en funktion som er blevet konstrueret ud fra samme fremgangsmåde som Initialize() funktionen ved at aflæse en graf der indeholder intervaller, der skal til for at tilgå sensoren.

\begin{figure}[h!]
  \centering
  \includegraphics[width=1\textwidth]{figures/write_byte.png}
  \caption{writeByte() arduino kode.}
  \label{write_byte}
\end{figure}
\fxnote{set firkant eller noget rundt om alle vores kodeeksempler så de ikke går i et med teksten omkring}

writeByte 