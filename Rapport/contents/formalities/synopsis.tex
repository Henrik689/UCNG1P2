Dette projekt omhandler design og konstruktion af en Hi-Fi forstærker med klasse G effekttrin. I forbindelse med dette tages der fat i standarder for Hi-Fi forstærkere og lyd.\\

Forskellige former for audio forstærkere analyseres, herunder udvalgte klasser af udgangstrin. Det vælges at forstærkeren får to indgangssignaler fra henholdsvis CD-afspiller og line signal (målt for iPhone 4s).\\

Der designes forforstærker, tonekontrol, volumenkontrol, effekttrin og strømforsyning. På disse er der blevet udført beregninger, simuleringer samt dimensionering af komponenter. Det har ikke været muligt at overholde kravspecifikationen, idet kun dele af systemet er blevet konstrueret og testet.





%Rapporten omhandler udvikling af en hi-fi forstærker som er strømbesparende, og giver en forbedret nyttevirkning i forhold til en klassiske AB forstærker.  For at reducere effektforbruget udvikles forstærkeren med et klasse G udgangstrin, dette grunder i den lave nyttevirkning som forekommer ved lav udgangseffekt.
%\\\\
%Der arbejdes med et indgangssignal, strømforsyning, tonekontrol, volumenkontrol og effekt udgangstrin i denne rapport.
%\\\\
%Der bruges en lånt strømforsyning med 2 udgangsspændinger, så der lægges vægt på udviklingen af de forskellige elementer i en forstærker. For de forskellige elementer som forstærkeren består af, tages der fat i standarder for hi-fi forstærkere og lyd som udgangspunkt, så de opstillede krav overholdes.
%\\\\
%Rapporten beskriver processen igennem udviklingen af en prototype med beregninger, målinger, laboratorieforsøg og simuleringer, og kommer frem til en forbedret forstærker med god lydkvalitet og med høj nyttevirkning ved lav udgangseffekt.