Som et led i arbejd processen med udvikling af systemet er der blevet arbejdet på hvordan det system kan videre udvikles til at have mere funktion i forhold til at dette system skulle anvendes i en virkeligheds situation.

\section{Data logning}
Den første del af videre udviklingen består i at gøre systemet i stand til ikke bare at kunne visse de målte data og anden beregning der bliver foretaget, men at logge det over tid så det både kan bruges til at logfører markante ændringer over tid, samt at kunne fremvise det loggede data for en forbruger af systemet.

\subsection{Arduino [Hello world]\fixme{mangler bedre titel}}
For at gøre systemet i stand til at kunne logge de målte og beregnede data er der til dette projekt blevet valgt at bruge et Arduino Ethernet Shield for at finde arduinoen med internettet, samt der er blevet skrevet kode der gør det muligt for arduinoen at lægge data op på en SQL database på nettet.
\img{figures/ethernetshield.jpg}{Arduino Ethernet Shield. Foto af: http://tech-things.net/}{ethernetshield}{0.6}
program flowet med brugen af ethernet shieldet følger det samme flow vist i figur \ref{fase1flow}

\subsection{GUI hjemmeside}
\section{Lejligheds løsning}
\subsection{Radio moduler}
\subsection{Mesh netværk}
